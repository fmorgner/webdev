\section{C++ for the web}
\begin{frame}
  \begin{center}
    \huge C++ for the web
  \end{center}
\end{frame}
\begin{frame}
  \frametitle{}
  \begin{center}
    \includegraphics[width=6cm]{images/why}
    \\ And how?
  \end{center}
\end{frame}
\begin{frame}
  \frametitle{How}
  \begin{itemize}
    \item Full-stack framework
    \item Toolkit
    \item Roll your own
  \end{itemize}
\end{frame}
\begin{frame}
  \frametitle{The toolkit}
  \begin{itemize}
    \item \href{https://github.com/etr/libhttpserver}{libhttpserver}
    \item \href{https://github.com/open-source-parsers/jsoncpp}{jsoncpp}
    \item \href{https://github.com/hmartiro/redox}{redox}
    \item \href{https://github.com/no1msd/mstch}{mstch}
  \end{itemize}
\end{frame}
\begin{frame}[fragile]
  \frametitle{libhttpserver - the server}
  \begin{lstlisting}[language={[11]C++}]
  auto creator = create_webserver{80}.max_threads(5);
  auto server = webserver{creator};

  server.start(true);
  \end{lstlisting}
\end{frame}
\begin{frame}[fragile]
  \frametitle{libhttpserver - the resources}
  \begin{lstlisting}[language={[11]C++}]
  struct index : http_resource<index>
  {
    void render_GET(http_request const &,
    http_response * * const);
  };
  \end{lstlisting}
\end{frame}
\begin{frame}[fragile]
  \frametitle{libhttpserver - the resources (contd.)}
  \begin{lstlisting}[language={[11]C++}]
  void index::render_GET(http_request const & req,
  http_response * * const res)
  {
    auto builder = http_response_builder{"Hi!",
                                         200,
                                         "text/html"};

    * response = new http_response{builder};
  }
  \end{lstlisting}
\end{frame}
\begin{frame}[fragile]
  \frametitle{libhttpserver - the server with a resource}
  \begin{lstlisting}[language={[11]C++}]
  auto creator = create_webserver{80}.max_threads(5);
  auto server = webserver{creator};

  auto resource = index{};
  server.register_resource("/", &resource, true);

  server.start(true);
  \end{lstlisting}
\end{frame}
\begin{frame}
  \begin{itemize}
    \item raw pointers / new
    \item libmicrohttpd
    \item intuitive
  \end{itemize}
\end{frame}
\begin{frame}[fragile]
  \frametitle{jsoncpp - data serialization}
  \begin{lstlisting}[language={[11]C++}]
  using namespace std::literals;

  auto root = Json::Value{};

  root["text"] = "jsoncpp is easy"s;
  \end{lstlisting}
\end{frame}
\begin{frame}[fragile]
  \frametitle{jsoncpp - data serialization (contd.)}
  \begin{lstlisting}[language={[11]C++}]
  using namespace std::literals;

  Json::Value root{};

  root["foo"]["bar"] = "baz"s;
  \end{lstlisting}
\end{frame}
\begin{frame}[fragile]
  \frametitle{jsoncpp - data serialization (contd.)}
  \begin{lstlisting}[language=Javascript]
  {
    "foo" : {
      "bar" : "baz"
    }
  }
  \end{lstlisting}
\end{frame}
\begin{frame}[fragile]
  \frametitle{jsoncpp - data deserialization}
  \begin{lstlisting}[language={[11]C++}]
  using namespace std::literals;
 
  auto root = Json::Value{};

  auto input = "{\"text\": \"deserialization\"}"s;

  Json::Reader::parse(input, root, false);
  \end{lstlisting}
\end{frame}
\begin{frame}
  \begin{itemize}
    \item simple / intuitive
    \item can preserve comments
    \item Some traps (e.g strings)
  \end{itemize}
\end{frame}
\begin{frame}[fragile]
  \frametitle{redox}
  \begin{lstlisting}[language={[11]C++}]
  redox::Redox redis{};

  redis.connect();

  auto res = redis.commandSync<int>({"SISMEMBER",
                                     "users", id});
  \end{lstlisting}
\end{frame}
\begin{frame}[fragile]
  \frametitle{redox (contd.)}
  \begin{lstlisting}[language={[11]C++}]
  redox::Redox redis{};
  redox::Subscriber sub{};

  redis.connect();
  sub.connect();
  sub.subscribe("demo", [](auto const & top,
                           auto const & msg){
    // handle msg
  });
  \end{lstlisting}
\end{frame}
\begin{frame}
  \begin{itemize}
    \item no boost
    \item allows synchronous
    \item fairly new
  \end{itemize}
\end{frame}
\begin{frame}[fragile]
  \frametitle{mstch}
  \begin{lstlisting}[language={[11]C++}]
  using namespace std::literals;

  auto tmpl = "Hi {{name}}"s;
  auto data = mstch::map{{"name", "fmo"s}};

  auto result = mstch::render(tmpl, data);
  \end{lstlisting}
\end{frame}
\begin{frame}[fragile]
  \frametitle{mstch partials}
  \begin{lstlisting}[language={[11]C++}]
  using namespace std::literals;

  auto tmpl = "Hi {{>part}}"s;
  auto data = map{{"name", "fmo"s}};

  auto result = render(tmpl, data, {{"part",
                                     "{{name}}"}});
  \end{lstlisting}
\end{frame}
\begin{frame}[fragile]
  \frametitle{mstch - advanced templates}
  \begin{lstlisting}[language=html]
  <ul class="list-group">
    {{#shouts}}
    <li class="list-group-item">
      <h3 class="list-group-item-heading">{{text}}</h3>
    </li>
    {{/shouts}}
  </ul>
  \end{lstlisting}
\end{frame}
\begin{frame}[fragile]
  \frametitle{mstch - contd.}
  \begin{lstlisting}[language={[11]C++}]
  using namespace std::literals;

  auto tmpl = read_file("tmpl.html");
  auto hi = map{{"text", "Hi!"s}};
  auto bye = map{{"text", "Bye!"s}};
  auto data = array{hi, bye};

  auto result = render(tmpl, map{{"shouts", data}});
  \end{lstlisting}
\end{frame}
\begin{frame}[fragile]
  \frametitle{mstch - advanced templates}
  \begin{lstlisting}[language=html]
  <ul class="list-group">
    <li class="list-group-item">
      <h3 class="list-group-item-heading">Hi!</h3>
    </li>
    <li class="list-group-item">
      <h3 class="list-group-item-heading">Bye!</h3>
    </li>
  </ul>
  \end{lstlisting}
\end{frame}
\begin{frame}
  \begin{itemize}
    \item full mustache
    \item boost :(
    \item Some traps (e.g strings)
  \end{itemize}
\end{frame}
